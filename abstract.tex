\begin{abstract}
Human dectection is an important technique used in many other problems such as traffic control, movies' features extracting, etc. However, it's hard to detect the whole body because the lower part is usually obscured by other objects like bikes, cars, tables, etc. This motivates us to use R-CNN to detect upper-body of the target.  The dataset consists of 98 images. People in these images are from different ages and have different angles. This method gives a promising result: the average R value which shows accuracy rate is about 82\%. This method can be improved by using better dataset.
\end{abstract}