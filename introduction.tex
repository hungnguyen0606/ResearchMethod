\section{Introduction}
% no \IEEEPARstart
The increasing development of machine learning provide us more tools for different kind of detection problems. The biggest barrier of this type of problem is that one object can have different shapes, and their appearance may vary due to different angles of the camera. This is why these problems are still difficult despite the rapid development of computer hardwares.

One of the object detection problems is human detection in images. Human detection is an important step to solve other important problems such as pedestrian detection system in self-driving cars or surveillance camera systems. However, in most images we don't usually have a whole human body, because human bodies are mostly obscured by other objects like bikes, cars, tables, trees, etc. This is an obstacle that can dramatically decrease the accuracy of full-body detection algorithms. That's why the authors recommend detecting upper body only using Region-based Convolutional Neural Networks.

There are many methods used for human detection, including SVM with discrete Wavelet transform \cite{wavelet} and HOG\cite{hog}. However, these methods have some disadvantages. First of all, we must say that these methods are quite old. The original SVM algorithms is invented 1963 and non-linear classifiers are suggested in 1992.\cite{svm} The authors don't say that these methods are too old to be used, but after many years of using, they showed some weakness.
For example, if images have low contrast and unclear boundaries due to some factors such as light or low-quality cameras, the images will provide incomplete gradient information, the fact that dramatically decreases the accuracy of detectors. To get rid of these disadvantages, we can apply some image preprocessing techniques. However, each image has different characteristics, therefore requires different techniques to preserve its properties. 
%This prevents systems from detecting humans from images extracted from videos due to 
That is why we recommend a new method for this type of problem. The core idea of our method is using R-CNN trained with the dataset of humans' upper-body. To improve the accuracy, we pre-train the model using a large auxiliary dataset, ILSVRC, as recommended by \cite{rcnn}. After training with 98 images, the origin model achieves the accuracy of 70\%. Although this is not an impressive result for this problem. Many methods are proposed to solve this problem with higher accuracy, but few methods use R-CNN. By using a better dataset (with more images), we can improve the accuracy of the detector.

The rest of this paper is organized as follows. In section II, the authors describe dataset and methodology use to train and validate the model. In section III, we discuss about our model used in this paper. Section IV presents the experimental results and evaluations. The conclusions and future work are presented in section V. 